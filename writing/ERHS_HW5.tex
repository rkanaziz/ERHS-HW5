\documentclass[]{article}
\usepackage{lmodern}
\usepackage{amssymb,amsmath}
\usepackage{ifxetex,ifluatex}
\usepackage{fixltx2e} % provides \textsubscript
\ifnum 0\ifxetex 1\fi\ifluatex 1\fi=0 % if pdftex
  \usepackage[T1]{fontenc}
  \usepackage[utf8]{inputenc}
\else % if luatex or xelatex
  \ifxetex
    \usepackage{mathspec}
  \else
    \usepackage{fontspec}
  \fi
  \defaultfontfeatures{Ligatures=TeX,Scale=MatchLowercase}
\fi
% use upquote if available, for straight quotes in verbatim environments
\IfFileExists{upquote.sty}{\usepackage{upquote}}{}
% use microtype if available
\IfFileExists{microtype.sty}{%
\usepackage{microtype}
\UseMicrotypeSet[protrusion]{basicmath} % disable protrusion for tt fonts
}{}
\usepackage[margin=1in]{geometry}
\usepackage{hyperref}
\hypersetup{unicode=true,
            pdftitle={Homework \#5, Choice \#1},
            pdfauthor={Rachel Kanaziz},
            pdfborder={0 0 0},
            breaklinks=true}
\urlstyle{same}  % don't use monospace font for urls
\usepackage{color}
\usepackage{fancyvrb}
\newcommand{\VerbBar}{|}
\newcommand{\VERB}{\Verb[commandchars=\\\{\}]}
\DefineVerbatimEnvironment{Highlighting}{Verbatim}{commandchars=\\\{\}}
% Add ',fontsize=\small' for more characters per line
\usepackage{framed}
\definecolor{shadecolor}{RGB}{248,248,248}
\newenvironment{Shaded}{\begin{snugshade}}{\end{snugshade}}
\newcommand{\AlertTok}[1]{\textcolor[rgb]{0.94,0.16,0.16}{#1}}
\newcommand{\AnnotationTok}[1]{\textcolor[rgb]{0.56,0.35,0.01}{\textbf{\textit{#1}}}}
\newcommand{\AttributeTok}[1]{\textcolor[rgb]{0.77,0.63,0.00}{#1}}
\newcommand{\BaseNTok}[1]{\textcolor[rgb]{0.00,0.00,0.81}{#1}}
\newcommand{\BuiltInTok}[1]{#1}
\newcommand{\CharTok}[1]{\textcolor[rgb]{0.31,0.60,0.02}{#1}}
\newcommand{\CommentTok}[1]{\textcolor[rgb]{0.56,0.35,0.01}{\textit{#1}}}
\newcommand{\CommentVarTok}[1]{\textcolor[rgb]{0.56,0.35,0.01}{\textbf{\textit{#1}}}}
\newcommand{\ConstantTok}[1]{\textcolor[rgb]{0.00,0.00,0.00}{#1}}
\newcommand{\ControlFlowTok}[1]{\textcolor[rgb]{0.13,0.29,0.53}{\textbf{#1}}}
\newcommand{\DataTypeTok}[1]{\textcolor[rgb]{0.13,0.29,0.53}{#1}}
\newcommand{\DecValTok}[1]{\textcolor[rgb]{0.00,0.00,0.81}{#1}}
\newcommand{\DocumentationTok}[1]{\textcolor[rgb]{0.56,0.35,0.01}{\textbf{\textit{#1}}}}
\newcommand{\ErrorTok}[1]{\textcolor[rgb]{0.64,0.00,0.00}{\textbf{#1}}}
\newcommand{\ExtensionTok}[1]{#1}
\newcommand{\FloatTok}[1]{\textcolor[rgb]{0.00,0.00,0.81}{#1}}
\newcommand{\FunctionTok}[1]{\textcolor[rgb]{0.00,0.00,0.00}{#1}}
\newcommand{\ImportTok}[1]{#1}
\newcommand{\InformationTok}[1]{\textcolor[rgb]{0.56,0.35,0.01}{\textbf{\textit{#1}}}}
\newcommand{\KeywordTok}[1]{\textcolor[rgb]{0.13,0.29,0.53}{\textbf{#1}}}
\newcommand{\NormalTok}[1]{#1}
\newcommand{\OperatorTok}[1]{\textcolor[rgb]{0.81,0.36,0.00}{\textbf{#1}}}
\newcommand{\OtherTok}[1]{\textcolor[rgb]{0.56,0.35,0.01}{#1}}
\newcommand{\PreprocessorTok}[1]{\textcolor[rgb]{0.56,0.35,0.01}{\textit{#1}}}
\newcommand{\RegionMarkerTok}[1]{#1}
\newcommand{\SpecialCharTok}[1]{\textcolor[rgb]{0.00,0.00,0.00}{#1}}
\newcommand{\SpecialStringTok}[1]{\textcolor[rgb]{0.31,0.60,0.02}{#1}}
\newcommand{\StringTok}[1]{\textcolor[rgb]{0.31,0.60,0.02}{#1}}
\newcommand{\VariableTok}[1]{\textcolor[rgb]{0.00,0.00,0.00}{#1}}
\newcommand{\VerbatimStringTok}[1]{\textcolor[rgb]{0.31,0.60,0.02}{#1}}
\newcommand{\WarningTok}[1]{\textcolor[rgb]{0.56,0.35,0.01}{\textbf{\textit{#1}}}}
\usepackage{graphicx,grffile}
\makeatletter
\def\maxwidth{\ifdim\Gin@nat@width>\linewidth\linewidth\else\Gin@nat@width\fi}
\def\maxheight{\ifdim\Gin@nat@height>\textheight\textheight\else\Gin@nat@height\fi}
\makeatother
% Scale images if necessary, so that they will not overflow the page
% margins by default, and it is still possible to overwrite the defaults
% using explicit options in \includegraphics[width, height, ...]{}
\setkeys{Gin}{width=\maxwidth,height=\maxheight,keepaspectratio}
\IfFileExists{parskip.sty}{%
\usepackage{parskip}
}{% else
\setlength{\parindent}{0pt}
\setlength{\parskip}{6pt plus 2pt minus 1pt}
}
\setlength{\emergencystretch}{3em}  % prevent overfull lines
\providecommand{\tightlist}{%
  \setlength{\itemsep}{0pt}\setlength{\parskip}{0pt}}
\setcounter{secnumdepth}{0}
% Redefines (sub)paragraphs to behave more like sections
\ifx\paragraph\undefined\else
\let\oldparagraph\paragraph
\renewcommand{\paragraph}[1]{\oldparagraph{#1}\mbox{}}
\fi
\ifx\subparagraph\undefined\else
\let\oldsubparagraph\subparagraph
\renewcommand{\subparagraph}[1]{\oldsubparagraph{#1}\mbox{}}
\fi

%%% Use protect on footnotes to avoid problems with footnotes in titles
\let\rmarkdownfootnote\footnote%
\def\footnote{\protect\rmarkdownfootnote}

%%% Change title format to be more compact
\usepackage{titling}

% Create subtitle command for use in maketitle
\providecommand{\subtitle}[1]{
  \posttitle{
    \begin{center}\large#1\end{center}
    }
}

\setlength{\droptitle}{-2em}

  \title{Homework \#5, Choice \#1}
    \pretitle{\vspace{\droptitle}\centering\huge}
  \posttitle{\par}
    \author{Rachel Kanaziz}
    \preauthor{\centering\large\emph}
  \postauthor{\par}
      \predate{\centering\large\emph}
  \postdate{\par}
    \date{11/11/2019}


\begin{document}
\maketitle

\hypertarget{homicide-data}{%
\section{Homicide Data}\label{homicide-data}}

Read in the data.

\begin{Shaded}
\begin{Highlighting}[]
\NormalTok{homicide_raw <-}\StringTok{ }\KeywordTok{read_csv}\NormalTok{(}\StringTok{"../data/homicide-data.csv"}\NormalTok{)}
\end{Highlighting}
\end{Shaded}

\begin{verbatim}
## Parsed with column specification:
## cols(
##   uid = col_character(),
##   reported_date = col_double(),
##   victim_last = col_character(),
##   victim_first = col_character(),
##   victim_race = col_character(),
##   victim_age = col_character(),
##   victim_sex = col_character(),
##   city = col_character(),
##   state = col_character(),
##   lat = col_double(),
##   lon = col_double(),
##   disposition = col_character()
## )
\end{verbatim}

\hypertarget{clean-data}{%
\subsection{Clean Data}\label{clean-data}}

Filter the data for the city of Denver and select the categories that
will be used.

\begin{Shaded}
\begin{Highlighting}[]
\NormalTok{denver <-}\StringTok{ }\NormalTok{homicide_raw }\OperatorTok\StringTok{ }
\StringTok{  }\KeywordTok{filter}\NormalTok{(city }\OperatorTok{==}\StringTok{ "Denver"}\NormalTok{) }\OperatorTok\StringTok{ }
\StringTok{  }\KeywordTok{select}\NormalTok{(lat, lon, disposition, victim_race)}
\end{Highlighting}
\end{Shaded}

\hypertarget{map-boundaries}{%
\subsection{Map Boundaries}\label{map-boundaries}}

In this section, I used the zip codes for the Denver area to create
boundaries for my base map.

\begin{Shaded}
\begin{Highlighting}[]
\NormalTok{denver_zips <-}\StringTok{ }\KeywordTok{zctas}\NormalTok{(}\DataTypeTok{cb =} \OtherTok{TRUE}\NormalTok{, }\DataTypeTok{starts_with =} \KeywordTok{c}\NormalTok{(}\StringTok{"802"}\NormalTok{), }\DataTypeTok{class =} \StringTok{"sf"}\NormalTok{)}
\KeywordTok{plot}\NormalTok{(denver_zips)}
\end{Highlighting}
\end{Shaded}

\includegraphics{ERHS_HW5_files/figure-latex/zips-1.pdf}

\hypertarget{number-of-homicides-by-victim-race}{%
\subsection{Number of Homicides by Victim
Race}\label{number-of-homicides-by-victim-race}}

Here, I determined the three race groups (balck, white, hispanic) with
the highest number of victim homicides.

\begin{Shaded}
\begin{Highlighting}[]
\NormalTok{race <-}\StringTok{ }\NormalTok{denver }\OperatorTok\StringTok{ }
\StringTok{  }\KeywordTok{group_by}\NormalTok{(victim_race) }\OperatorTok\StringTok{ }
\StringTok{  }\KeywordTok{mutate}\NormalTok{(}\DataTypeTok{count=}\KeywordTok{n}\NormalTok{()) }\OperatorTok\StringTok{ }
\StringTok{  }\KeywordTok{arrange}\NormalTok{(}\KeywordTok{desc}\NormalTok{(count)) }\OperatorTok\StringTok{ }
\StringTok{  }\KeywordTok{ungroup}\NormalTok{()}

\NormalTok{denver2 <-}\StringTok{ }\NormalTok{denver }\OperatorTok
\StringTok{  }\KeywordTok{filter}\NormalTok{(victim_race }\OperatorTok{==}\StringTok{ }\KeywordTok{c}\NormalTok{(}\StringTok{"Black"}\NormalTok{, }\StringTok{"White"}\NormalTok{, }\StringTok{"Hispanic"}\NormalTok{)) }\OperatorTok\StringTok{ }
\StringTok{  }\KeywordTok{mutate}\NormalTok{(}\DataTypeTok{disposition =} \KeywordTok{factor}\NormalTok{(disposition, }\DataTypeTok{levels =} \KeywordTok{c}\NormalTok{(}\StringTok{"Closed without arrest"}\NormalTok{, }\StringTok{"Closed by arrest"}\NormalTok{, }\StringTok{"Open/No arrest"}\NormalTok{), }
                      \DataTypeTok{labels =} \KeywordTok{c}\NormalTok{(}\StringTok{"solved"}\NormalTok{, }\StringTok{"solved"}\NormalTok{, }\StringTok{"unsolved"}\NormalTok{)))}
\end{Highlighting}
\end{Shaded}

\hypertarget{create-the-map}{%
\subsection{Create the Map}\label{create-the-map}}

I created a map to show where the homicides in Denver have occured for
the three race groups with the highest number of victims. Then, I
faceted by the disposition of the case in order to create two stacked
maps.

\begin{Shaded}
\begin{Highlighting}[]
\NormalTok{denver_crs <-}\StringTok{ }\NormalTok{denver2 }\OperatorTok\StringTok{ }
\StringTok{  }\KeywordTok{filter}\NormalTok{(}\OperatorTok{!}\KeywordTok{is.na}\NormalTok{(lat)) }\OperatorTok\StringTok{ }
\StringTok{  }\KeywordTok{st_as_sf}\NormalTok{(}\DataTypeTok{coords =} \KeywordTok{c}\NormalTok{(}\StringTok{"lon"}\NormalTok{, }\StringTok{"lat"}\NormalTok{)) }\OperatorTok\StringTok{ }
\StringTok{  }\KeywordTok{st_set_crs}\NormalTok{(}\DecValTok{4269}\NormalTok{)}

\NormalTok{map <-}\StringTok{ }\KeywordTok{ggplot}\NormalTok{() }\OperatorTok{+}\StringTok{ }
\StringTok{  }\KeywordTok{geom_sf}\NormalTok{(}\DataTypeTok{data =}\NormalTok{ denver_zips, }\DataTypeTok{color =} \StringTok{"lightgray"}\NormalTok{) }\OperatorTok{+}\StringTok{ }
\StringTok{  }\KeywordTok{geom_sf}\NormalTok{(}\DataTypeTok{data =}\NormalTok{ denver_crs, }\KeywordTok{aes}\NormalTok{(}\DataTypeTok{color =} \KeywordTok{factor}\NormalTok{(victim_race)), }\DataTypeTok{show.legend =} \StringTok{'point'}\NormalTok{) }\OperatorTok{+}
\StringTok{  }\KeywordTok{ggtitle}\NormalTok{(}\StringTok{"Homicides in Denver, CO"}\NormalTok{) }\OperatorTok{+}
\StringTok{  }\KeywordTok{scale_color_manual}\NormalTok{(}\DataTypeTok{values =} \KeywordTok{c}\NormalTok{(}\StringTok{"Black"}\NormalTok{ =}\StringTok{ "black"}\NormalTok{, }\StringTok{"Hispanic"}\NormalTok{ =}\StringTok{ "red"}\NormalTok{, }\StringTok{"White"}\NormalTok{ =}\StringTok{ "yellow"}\NormalTok{),}
                     \DataTypeTok{labels =} \KeywordTok{c}\NormalTok{(}\StringTok{"Black"}\NormalTok{, }\StringTok{"Hispanic"}\NormalTok{, }\StringTok{"White"}\NormalTok{),}
                     \DataTypeTok{name =} \StringTok{"Victim Race"}\NormalTok{) }\OperatorTok{+}
\StringTok{  }\KeywordTok{labs}\NormalTok{(}\DataTypeTok{x =} \StringTok{"Longitude"}\NormalTok{, }\DataTypeTok{y =} \StringTok{"Latitude"}\NormalTok{) }\OperatorTok{+}
\StringTok{  }\KeywordTok{facet_wrap}\NormalTok{(}\OperatorTok{~}\StringTok{ }\NormalTok{disposition, }\DataTypeTok{ncol =} \DecValTok{1}\NormalTok{) }\OperatorTok{+}
\StringTok{  }\KeywordTok{theme}\NormalTok{(}\DataTypeTok{plot.title =} \KeywordTok{element_text}\NormalTok{(}\DataTypeTok{hjust =} \FloatTok{0.5}\NormalTok{), }\DataTypeTok{axis.text.x =} \KeywordTok{element_text}\NormalTok{(}\DataTypeTok{angle =} \DecValTok{90}\NormalTok{, }\DataTypeTok{hjust =} \DecValTok{1}\NormalTok{))}

\NormalTok{map}
\end{Highlighting}
\end{Shaded}

\includegraphics{ERHS_HW5_files/figure-latex/map-1.pdf}


\end{document}
